\documentclass{article}
\usepackage[utf8]{inputenc}
\usepackage{enumitem}
\usepackage{natbib}

\title{Mobile Development, Native and Cross-Platform Applications, Design Patterns}
\author{Beltran Ramirez Erik Alfonso}
\date{03/01/24}

\begin{document}

\maketitle

\section{Fundamentals of Mobile Development}

Mobile development is a branch that focuses on creating applications for mobile devices such as phones and tablets. To fully understand its fundamentals, we need to consider the following:

\begin{itemize}[label=--]
    \item \textbf{Mobile Platforms:} There are various mobile platforms, with Android and iOS being the most prominent. Each has its own ecosystem and set of development tools.
    \item \textbf{Specific Operating Systems:} In-depth knowledge of the specific operating systems of each platform is crucial. This involves understanding the unique features of Android and iOS, as well as best practices to leverage them.
    \item \textbf{Technologies and Tools:} Familiarity with mobile development technologies and tools, such as Integrated Development Environments (IDE), frameworks, and Software Development Kits (SDKs).
\end{itemize}

\section{Native and Cross-Platform Applications}

Mobile application development is categorized into two main types: native and cross-platform.

\subsection{Native Applications}

Native applications are designed and developed specifically for a particular platform, maximizing the use of platform-specific features and functionalities. In the case of Android, languages like Java or Kotlin are used, while Swift and Objective-C are common for iOS.

\subsection{Cross-Platform Applications}

In contrast, cross-platform applications are developed to run on multiple platforms. This is achieved using technologies like React Native, Flutter, or Xamarin, enabling more efficient development by sharing code across platforms.

\section{Design Patterns for Mobile Applications}

Design patterns are reusable strategies that provide solutions to common problems in software development. Some fundamental design patterns for mobile applications include:

\begin{itemize}[label=--]
    \item \textbf{MVC (Model-View-Controller):} This pattern divides the application into three main components, facilitating modularity and code maintenance.
    
    \item \textbf{MVVM (Model-View-ViewModel):} Similar to MVC, but with an additional layer (ViewModel) that manages logic and the state of the user interface. This approach is especially useful in mobile development environments.
    
    \item \textbf{Singleton Pattern:} Ensures that a class has only one instance and provides a global access point to that instance. This is beneficial for efficiently managing shared resources.
    
    \item \textbf{Observer Pattern:} Defines a one-to-many dependency between objects, so that when one object changes state, all its dependents are notified. This pattern is valuable in situations where updates to multiple components are required.
\end{itemize}

\section{Sources}

\begin{enumerate}
    \item Bachi. (2023, August 28). \textit{Discover everything about mobile development: fundamentals, tips, and trends}. FelinoHost.
    https://felinohost.com/desarrollo-movil/descubre-todo-sobre-el-desarrollo-movil-fundamentos-consejos-y-tendencias/
    
    \item Zapater, S. (2022, May 28). \textit{Hybrid or Native App: Differences and Examples}. Hiberus Blog.
    https://www.hiberus.com/crecemos-contigo/app-hibrida-o-nativa/
    
    \item Txema Rodríguez. (2011, April 2). \textit{Android Patterns: Design patterns for developing Android applications.}.
    https://www.genbeta.com/desarrollo/android-patterns-patrones-de-diseno-para-desarrollar-aplicaciones-android
\end{enumerate}

\end{document}
